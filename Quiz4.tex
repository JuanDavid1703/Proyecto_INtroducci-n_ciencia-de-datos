\documentclass[10pt]{article}
\textheight=23cm
\textwidth=18cm
\topmargin=-1cm
\oddsidemargin=-1cm
\parindent=0cm

\usepackage[latin1]{inputenc}
\usepackage[spanish]{babel}
\usepackage{amsmath}
\usepackage{amsfonts}
\usepackage{amssymb}
\usepackage{graphicx}
\usepackage{graphics}
\DeclareGraphicsExtensions{.png,.pdf,.jpg,.gif}
%%%%%%%%%%%%%%%%%%%%%%%%%%%%%%%%%%%%%%%%%%%%%%%%%%%%%%%%%%%%%
\def\BB{\mathbb{B}}
\def \limx{n \rightarrow \infty}
\def\QQ{\mathbb{Q}}
\def\ZZ{\mathbb{Z}}
\def\RR{\mathbb{R}}
\def\NN{\mathbb{N}}
\def\FF{\mathbb{F}}
\newtheorem{Def}{Definici�n}
\newtheorem{Teo}{Teorema}
\newtheorem{Dem}{Demostraci�n}
\newtheorem{Ejem}{Ejemplo}
\newtheorem{Sol}{Soluci�n}



\title{Quiz \# 4 Parte B}
\author{Juan David Leal Campuzano\\
�ngela Mar�a Arboleda\\
Juan Fernando Herrera Santana 
}
\date{}

\begin{document}

\maketitle

\begin{enumerate}
%%%%%%%%%%%%%%%%%%%%%%%%%%%%%%%%%%%%%%%%%%%%%%%%%%%%%%%%%%%%%%%

    \item (Valor 4 puntos) Sea $(\Omega ,\FF,P)=((0,1),B(\RR),\lambda_{(0,1)})$, sea $X_n:= n^2 \chi_{(0,\frac{1}{n})}$.\\
    Verificar que:
    
    $$E(\liminf_{x\rightarrow \infty}{X_n})<\liminf_{x\rightarrow \infty}E(X_n)$$
    
    Sol: Podemos observar que $E(X_n)=n^2 \lambda(0,\frac{1}{n})=n$, si aplicamos el l�mite a esta expresi�n se obtiene:
    $$\lim_{x\rightarrow \infty} E(X_n)=\lim_{x\rightarrow \infty}n^2 \lambda(0,\frac{1}{n})=\lim_{x\rightarrow \infty}n=\infty$$
    
    adem�s $\lim_{x\rightarrow \infty}X_n=\lim_{x\rightarrow \infty}n^2\chi_{(0,\frac{1}{n})}=0$.\\ 
    Por otro lado si el l�mite de una sucesi�n $X_n$ existe entonces $\limsup X_n=\liminf X_n$, como $\lim_{x\rightarrow \infty} E(X_n)$ y $\lim_{x\rightarrow \infty}X_n$ existen entonces 
    $\limsup E(X_n)=\liminf E(X_n)=\infty$ y $\limsup X_n=\liminf X_n = 0$, as�
    $$0=E(\liminf X_n)<\liminf E(X_n)=\infty$$
    
    Ahora esto tambi�n se cumple para el l�mite superior
    
    $$0=E(\limsup X_n)<\limsup E(X_n)=\infty$$
    
    Esto no contradice el corolario consecuente del lema de Fatou porque no existe una variable aleatoria $Z$ tal que $X_n\leq Z$ para todo $n\geq 1$ es decir $X_n$ no es acotada para todo $n\geq 1$
    
%%%%%%%%%%%%%%%%%%%%%%%%%%%%%%%%%%%%%%%%%%%%%%%%%%%%%%%%%%%%%    
    \item (Valor 6 puntos) Sea $(\Omega ,\FF,P)=((0,1),B(\RR),\lambda_{(0,1)})$, suponga
    $$X_n(w)=\begin{cases}
        2n^2& \frac{1}{2n}\leq w \leq \frac{1}{n}\\
        0& O. W.
    \end{cases}$$
    
    \begin{enumerate}
     \item �Satisface $X_n$ el lema de Fatou? Explicar\\
     Si lo satsiface dado que:
     \begin{itemize}
         \item $\lim_{n \rightarrow \infty}{X_n}=0$ y dado esto se sabe que  $\limsup X_n=\liminf X_n=0$
         \item $\lim_{n \rightarrow \infty}{E(X_n)}=\lim_{n \rightarrow \infty}{2n^2 \lambda(\frac{1}{2n}, \frac{1}{n})}=\lim_{n \rightarrow \infty}{n}=\infty$, y por la misma argumentaci�n anterior $\limsup E(X_n)=\liminf E(X_n)=\infty$
         \item Teniendo en cuenta lo anterior $E(\liminf X_n)=0$ y as� se cumple la desigualdad  
        $$0=E(\liminf X_n)<\liminf E(X_n)=\infty$$
        \end{itemize}
    \item �Satisface $X_n$ el teorema de convergencia dominada? Explicar\\
        Observando que $\lim E(X_n)=\infty$ y que $X_n\rightarrow X=0$ entonces 
        $$\infty=\lim E(X_n)\neq E(X)=0$$
        No lo cumple
    \end{enumerate}

%%%%%%%%%%%%%%%%%%%%%%%%%%%%%%%%%%%%%%%%%%%%%%%%%%%%%%%%%%%%
    
    \item ( Valor 5 puntos) Sea $(X_n)_{n\geq 1}$ una sucesi�n de variables aleatorias tales que $-a<X_1<0$ para una constante $0<a<\infty$ y con $X_m\leq X_n$ para $m\leq n$. Demostrar que:
    $$\lim_{\limx}E(X_n)=E(\lim_{\limx}X_n)$$
    
    Sol: Hay dos casos,el primero en el que $X_n$ sea acoatada y en el caso de que no lo sea.
    \begin{itemize}
        \item Suponga que $X_n$ es no acotada, por ende como $X_n$ es creciente entonces $X_n \rightarrow \infty$ y as� $E(X_n) \rightarrow \infty$ dado que $E(\infty)=\infty$, as� se cumplir�a que $\lim E(X_n)=E(\lim X_n)$
        \item Ahora suponga que $X_n$ es una sucesi�n acotada, es decir que existe $K\in \RR$ tal que $-a< X_n\leq K$ para todo $n$, dado que es creciente entonces converge,entonces :
        $$E(X_n)\begin{cases}
            -E(X_n) & \mbox{ si } 0<-X_n<a\\
            E(X_n) & \mbox{ si } 0<X_n\\
        \end{cases}$$
        
        entonces por el teorema de convergencia mon�tonoa 
        $$\lim E(X_n)=E(\lim X_n)$$
    \end{itemize}

%%%%%%%%%%%%%%%%%%%%%%%%%%%%%%%%%%%%%%%%%%%%%%%%%%%%%%%%%%%%%%%

    \item ( Valor 10 puntos) Demostrar que para cualquier variable aleatoria no
negativa $X$ se satisface que:
\begin{itemize}
    \item $E(x)+E(\frac{1}{x})\geq 2$\\
    Dado que X es una variable no negativa entonces $E(X)\geq 0$, as� definase la variable $Z=\frac{(X-1)^2}{X}$, se tiene que $Z$ es no negativa dado que $(X-1)^2$ es no negativa y $X$ por hip�tesis tambi�n, as� entonces $E(Z)\geq 0$, partiendo de esto se da que:
    \begin{align*}
        E(Z)\geq& 0\\
        E(\frac{(X-1)^2}{X})\geq& 0\\
        E(\frac{X^2-2x+1}{X})\geq& 0\\
        E(\frac{X^2+1}{X}-2)\geq& 0\\
        E(\frac{X^2+1}{X})-E(2)\geq& 0\\
        E(\frac{X^2+1}{X})\geq& 2\\
        E(X+\frac{1}{X})\geq& 2\\
        E(X)+E(\frac{1}{X})\geq& 2\\
        &\blacksquare
    \end{align*}
    
    
    \item $E(max(X,\frac{1}{X}))\geq 1$\\
    Primero ve�se que la funci�n m�ximo se puede escribir de la siguiente manera:
    $$max(a,b)=\frac{a+b+|a-b|}{2}$$
    entonces la funci�n m�ximo se define de la siguiente manera
    
    $$max(a,b)=\begin{cases}
        \frac{a+b+a-b}{2}& \mbox{ si } a-b>0\\
        \frac{a+b-a+b}{2}& \mbox{ si } a-b<0\\
    \end{cases}$$
    $$max(a,b)=\begin{cases}
        a& \mbox{ si } a>b\\
        b& \mbox{ si } a<b\\
    \end{cases}$$
    
    Ahora def�nase la variable aleatoria $W=\frac{(X-1)^2+|X^2-1|}{2X}$, dado que 
    $(X-1)^2$, $|X^2-1|$, $X$, son no negativas entonces $E(W)$ es no negativo, ahora
    
    \begin{align*}
        E(W)\geq& 0\\
        E(\frac{(X-1)^2+|X^2-1|}{2X})\geq& 0\\
        E(\frac{X^2-2X+1+|X^2-1|}{2X})\geq& 0\\
        E(\frac{X^2+1+|X^2-1|}{2X})- E(1)\geq& 0\\
        E(\frac{\frac{X^2+1}{X}+\frac{|X^2-1|}{X}}{2})\geq& 1\\
        E(\frac{X+\frac{1}{X}+|X-\frac{1}{X}|}{2})\geq& 1\\
        E(max(X,\frac{1}{X}))\geq& 1\\
        &\blacksquare
    \end{align*}
\end{itemize}

%%%%%%%%%%%%%%%%%%%%%%%%%%%%%%%%%%%%%%%%%%%%%%%%%%%%%%%%%%%%%%

    \item ( Valor 15 puntos) Se dice que una variable aleatoria X tiene distribuci�n Beta con par�metro $\alpha>0$ y $\beta>0$ y se escribe $\overset{d}{\approx}BETA(\alpha,\beta)$ si su finci�n de densidad esta definidapor:
    $$f(x):=\frac{1}{B(\alpha, \beta)}x^{\alpha-1}(1-x)^{\beta-1}\mbox{  si  } 0<x<1$$
    y $f(x):=0 \mbox{ en otro caso}$\\
    
    donde $B(\alpha, \beta)=\frac{\Gamma(\alpha)\Gamma(\beta)}{\Gamma(\alpha+\beta)}$, demostrar que:
    
    \begin{enumerate}
        \item $$E(X^k)=\prod_{j=0}^{k-1}\frac{\alpha +j}{\alpha+\beta+j} \mbox{, } k=1,2,3 ...$$
        %%%%%%fraci�n    \frac{arriba}{abajo}
        %%%%%%  alpha \alpha,  beta \beta, \Gamma, \rho, \lambda, e^{exponente}
        %%%%%% integral   \int_{liminf}^{limsup}{___} , inversa h^{-1}
        %%%%%%%   absoluto | |, 
        %%%%% Para alinear ecuaciones \begin{align*}   \end{align*}
        %%%%% Para escribir 
        %%%%%% exponente _^{  }     subindice    -_{  }

\begin{align*}
E [X^{k}]=& \int_{0}^{1} x^{k} \frac{\Gamma(\alpha+\beta)}{\Gamma(\alpha) \Gamma(\beta)} x^{\alpha-1}(1-x)^{\beta-1} d x=\frac{\Gamma(\alpha+\beta)}{\Gamma(\alpha) \Gamma(\beta)} \int_{0}^{1} x^{k+\alpha-1}(1-x)^{\beta-1} d x \\
=& \frac{\Gamma(\alpha+k) \Gamma(\beta)}{\Gamma(k+\alpha+\beta)} \frac{\Gamma(\alpha+\beta)}{\Gamma(\alpha) \Gamma(\beta)} \underbrace{\int_{0}^{1} \frac{\Gamma(k+\alpha+\beta)}{\Gamma(\alpha+k) \Gamma(\beta)} x^{(k+\alpha)-1}(1-x)^{\beta-1} d x}_{\text {funci�n de densidad de } B(k+\alpha, \beta) \text { que integra } 1} \\
=& \frac{\Gamma(\alpha+k)}{\Gamma(\alpha)} \frac{\Gamma(\alpha+\beta)}{\Gamma(\alpha+\beta+k)}=\frac{(\alpha+k-1)(\alpha+k-2) \ldots \alpha \Gamma(\alpha)}{\Gamma(\alpha)} \\
& \times \frac{\Gamma(\alpha+\beta)}{(\alpha+\beta+k-1)(\alpha+\beta+k-2) \ldots(\alpha+\beta) \Gamma(\alpha+\beta)} \\
=& \frac{(\alpha+k-1) \ldots  \alpha}{(\alpha+\beta+k-1) \ldots(\alpha+\beta)}=\prod_{j=0}^{k-1}\frac{\alpha +j}{\alpha+\beta+j} \mbox{, } k=1,2,3 ... 
\end{align*}

Por lo tanto, para obtener la media, se tiene k=1 y entonces,
$$
 E[X]=\frac{\alpha}{\alpha+\beta}
$$
Para el segundo momento, se tiene k=2 y entonces,
$$
 E[X^{2}]=\frac{(\alpha+1) \alpha}{(\alpha+\beta+1)(\alpha+\beta)}
$$
y para la varianza se tiene que
$$
\operatorname{Var}(X)=E[X^{2}]-E[X]^{2}=\frac{\alpha \beta}{(\alpha+\beta)^{2}(\alpha+\beta+1)}
$$
        
        \item $$E(X)=\frac{\alpha}{\alpha+\beta}~~~~~~Var(X)=\frac{\alpha \beta}{(\alpha+\beta+1)(\alpha + \beta)^2}$$
        
        \begin{align*}
        E(X^k)=&\frac{1}{B(\alpha,\beta)}\int_{0}^{1}{(x^{k+\alpha-1}(1-x)^{b-1}}dx\\
        =&\frac{B(\alpha+k,\beta)}{B(\alpha,\beta)}\\
        =& \frac{\Gamma(\alpha+k)\Gamma(\alpha+\beta)}{\Gamma(\alpha+k+\beta)\Gamma(\alpha)}\\
        \end{align*}
        
        Usando lo anterior procedemos a hallar la esperanza y la varianza de la funci�n $B(\alpha, \beta)$: 
        
        Esperanza
        \begin{align*}
            E(X) =& \frac{\Gamma(\alpha+1)\Gamma(\alpha+\beta)}{\Gamma(\alpha+1+\beta)\Gamma(a)}\\
            =& \frac{\alpha \Gamma(\alpha)\Gamma(\alpha+\beta)}{(\alpha +\beta)\Gamma(\alpha+\beta)\Gamma(\alpha)}\\
            =&\frac{\alpha}{\alpha + \beta}
        \end{align*}
        
        
        \begin{align*}
            E(X^2) =& \frac{\Gamma(\alpha+2)\Gamma(\alpha+\beta)}{\Gamma(\alpha+2+\beta)\Gamma(\alpha)}\\
            =&\frac{(\alpha+1)\alpha}{(\alpha+\beta+1)(\alpha +\beta)}
        \end{align*}
        
        Varianza:
        \\
        \begin{align*}
            Var(X)=& E(X^2)-(E(X))^2\\
            =&\frac{(\alpha+1)\alpha}{(\alpha+\beta+1)(\alpha +\beta)} - (\frac{\alpha}{\alpha + \beta})^2\\
            =&\frac{(\alpha+1)(\alpha +\beta)\alpha-\alpha^2(\alpha+\beta+1)}{(\alpha+\beta+1)(\alpha +\beta)^2}\\
            =&\frac{(\alpha^2)(\alpha) +(\alpha^2)(\beta)+\alpha^2+\alpha\beta-(\alpha^2)(\alpha)-(\alpha^2)(\beta)-\alpha^2}{(\alpha+\beta+1)(\alpha +\beta)^2}\\
            =&\frac{\alpha\beta}{(\alpha+\beta+1)(\alpha +\beta)^2}\\
        \end{align*}
        
        
        \item Si $V$ y $W$ son variables aleatorias independientes con $V\overset{d}{\approx}\Gamma(\alpha, \lambda)$ y $W\overset{d}{\approx}\Gamma(\beta, \lambda)$, entonces la variable 
        $$\frac{V}{W+V}\overset{d}{\approx} BETA(\alpha,\beta)$$
        
        Considerando la transformaci�n definida por:
        
        $$h(V,W)=(V+W,\frac{V}{V+W})$$
        Se puede observar que $h(W,V):(0,\infty)\times(0,\infty)\longrightarrow (0,\infty)\times (0,1)$
        \begin{align*}
            \frac{V}{V+W}=Y & ~~~ & V+W=X\\
            \frac{V}{X}=Y & ~~~ & V+W=X\\
            V=XY & ~~~ & V+W=X\\
            V=XY & ~~~ & W=X-XY\\
        \end{align*}
        
    As�    
        $$h^{-1}=(XY,X-XY)$$
        
        \begin{align*}
            J(h^{-1}(X,Y))= &\begin{array}{| c  c|}
            Y & X \\
             1-Y & -X 
        \end{array}\\
        =& |-YX-X+XY|\\
        =& X\\
        \end{align*}
     
     Como las variables aleatorias son independientes entonces $f_{V,W}=f_V(v)\cdot f_W(w)$:
     $$f_{V,W}(v,w)=\frac{\lambda^{\alpha_1+\alpha_2}}{\Gamma(\alpha_1)\Gamma(\alpha_2)}e^{-\lambda(w+v)}v^{\alpha_1-1}w^{\alpha_2-1}\cdot 1_{(0,\infty)}(v)\cdot 1_{(0,\infty)}(w)$$
     
     
     Por el toerema de tranformaci�n se tiene que 
     
     $$f_{X,Y}(x,y)=J(h^{-1}(X,Y))f_{V,W}(XY,X-XY)$$
     
     \begin{align*}
    f_{X,Y}(x,y)= & \frac{x\lambda^{\alpha_1+\alpha_2}}{\Gamma(\alpha_1)\Gamma(\alpha_2)}e^{-\lambda x}(xy)^{\alpha_1-1}(x-xy)^{\alpha_2-1}\cdot I_V(x,y)\\
         =&\left [ \frac{\lambda^{\alpha_1+\alpha_2}}{\Gamma(\alpha_1+\alpha_2)}x^{\alpha_1+\alpha_2-1}e^{-\lambda x}\cdot 1_{(0,\infty)}(x) \right ]
         \left [ \frac{\Gamma(\alpha_1+\alpha_2)}{\Gamma(\alpha_1)\Gamma(\alpha_2)}y^{\alpha_1+-1}(1-y)^{\alpha_2-1}\cdot 1_{(0,1)}(y) \right ]
     \end{align*}
    Si se toma a:
    $$g_X(x)=\frac{\lambda^{\alpha_1+\alpha_2}}{\Gamma(\alpha_1+\alpha_2)}x^{\alpha_1+\alpha_2-1}e^{-\lambda x}\cdot 1_{(0,\infty)}(x)$$
    $$g_Y(y)= \frac{\Gamma(\alpha_1+\alpha_2)}{\Gamma(\alpha_1)\Gamma(\alpha_2)}y^{\alpha_1-1}(1-y)^{\alpha_2-1}\cdot 1_{(0,1)}(y)$$
    
    se puede observar que $X\overset{d}{\approx}\Gamma(\alpha_1+\alpha_2,\lambda)$ y que 
    $Y\overset{d}{\approx} BETA(\alpha_1,\alpha_2)$ 
    \end{enumerate}

%%%%%%%%%%%%%%%%%%%%%%%%%%%%%%%%%%%%%%%%%%%%%%%%%%%%%%

    \item ( Valor 10 puntos) Sean $X $ y $Y$ variables aleatorias
    independientes e igualmente distribuidas con distribuci�n
    uniforme sobre el intervalo $(0, 2)$: Calcular:
    \begin{itemize}
        \item $P(X\geq 1 | X+Y \leq 3)$
        \begin{align*}
            P(X\geq 1 | X+Y \leq 3)=&\cfrac{P(X\geq 1 ~~and~~ X+Y \leq 3)}{P(X+Y \leq 3)}
        \end{align*}
        Veamos la distribuci�n de la variable $W=X+Y$\\
        
        Sea $W=X+Y$ y $V=X$, as� $h(W,X)=(V,W-V)$ y 
        $$J(h(W,X))=-1$$
        As� la funci�n conjunta es:
        
        $$f_{W~V}(w,v)=|J(h(w,v))|f_{V~Y}(v,w-v)$$\\
        $$f_{W~V}(w,v)=\begin{cases}
            \frac{1}{4} & 0<v<2~~0<w-v<2\\
            0 & O. W.\\
            \end{cases}$$
        $$f_{W~V}(w,v)=\begin{cases}
            \frac{1}{4} & 0<x<2~v<w<v+2\\
            0 & O. W.\\
            \end{cases}$$
        
        Para hallar $f_W(w)$ se tiene que si $0<w-v<2$ entonces $w-2<v<w$:
        $$f_{W}(w)=\begin{cases}
            &\int_{0}^{w}{\frac{dv}{4}}~~~~~~0<w<2\\
            &\int_{w-2}^{2}{\frac{dv}{4}}~~~~~~2<w<4\\
            \end{cases}$$
        $$f_{W}(w)=\begin{cases}
            =&\frac{w}{4}~~~~~~0<w<2\\
            & 1-\frac{w}{4}~~~~~~2<w<4\\
        \end{cases}$$
        
        Como $X$ e $Y$ son independientes entonces $X$ y $W$ tambi�n los son por ende la funci�n de probabilidad conjunta de $W$ y $X$ es el producto de las funciones de probabilidad univariadas
        
        $$f_{W~x}(w,x)=\begin{cases}
            \frac{w}{8} & 0<x<2~~0<w<2\\
            \frac{1}{2}-\frac{w}{8} & 0<x<2~~2<w<4\\
            0 & O.~~W.
        \end{cases}$$
        
        Devolvi�ndose a la probabilidade condicionada se da que 
        
        \begin{align*}
            P(X\geq 1 | X+Y \leq 3)=&\cfrac{P(X\geq 1 ~~and~~ X+Y \leq 3)}{P(X+Y \leq 3)}\\
            =&\cfrac{\int_{1}^{2}{\int_{0}^{2}{\frac{w}{8}dw}dx}+\int_{1}^{2}{\int_{2}^{3}{\frac{1}{2}-\frac{w}{8}dw}dx}}{\int_{0}^{2}{\frac{w}{4}dw}+\int_{2}^{3}{1-\frac{w}{4}}}\\
            =&\cfrac{7/16}{7/8}\\
            =&\cfrac{1}{2}
        \end{align*}
        
        \item $E(X|X+Y\leq 3)$
        \begin{align*}
            E(X|X+Y\leq 3)=&\int_0^2{\int_0^2{\frac{xw}{8}}dwdx}+\int_0^2{\int_2^2{\frac{4x-xw}{8}}dwdx}\\
            =&\int_0^2{xdx}\int_0^2{\frac{xw}{8}dw}+\int_0^2{\int_2^2{xdx}\frac{4-w}{8}dw}\\
            =&2(1/4)+2(3/8)\\
            =7/8
        \end{align*}
        
    \end{itemize}
    
\end{enumerate}

\end{document}